% document class settings
\documentclass[12pt, a4paper]{article}

% encoding
\usepackage[utf8]{inputenc}
\usepackage[T1]{fontenc}

% opmaak
\usepackage{microtype}
\usepackage{XCharter}
\usepackage[T1]{eulervm}
\usepackage[margin=2.5cm]{geometry}

\usepackage{titlesec}

% regelafstand
\usepackage{setspace}
\onehalfspacing%

% header & footer
\usepackage{fancyhdr}
\pagestyle{fancy}
\fancyhead[L]{\small\itshape~Introduction to Stata Programming}%
\fancyhead[R]{\small\itshape\leftmark~}%
\fancyfoot[C]{\thepage}%
\setlength{\headheight}{14pt}%
\renewcommand{\sectionmark}[1]{%
  \markboth{#1}{}}%

% bibliography
\usepackage[style=apa,natbib]{biblatex}
\addbibresource{references.bib}


% hyperlinks
\usepackage{hyperref}
\hypersetup{
  colorlinks=true,
  allcolors=black,
  linkcolor=blue,
  urlcolor=blue,
  citecolor=blue,
  linktocpage=true,
  hypertexnames=false
}
% easy references to tables/figures etc, not sure yet if necessary
\usepackage[capitalise,noabbrev]{cleveref}

% less "hanging" text at bottom or top of pages
\widowpenalty=10000
\clubpenalty=10000


\begin{document}

\begin{titlepage}\thispagestyle{empty}
    \begin{center}
        \vspace*{0.5cm}
        \LARGE
        Introduction to Stata Programming\\
        \Large
        Erasmus Thesis Project\\
        \vspace{1cm}
        \large
        Armin Hoendervangers\\
        AEclipse ETP Committee\\

        \vspace{2cm}
        Current version: \today\footnote{This is a work in progress, so the document will have incomplete sections and may contain mistakes.}

        \href{https://github.com/Ahvns/ETPreader/raw/main/Introduction%20to%20Stata%20Programming.pdf}{Click here for latest version}

        \vfill
    \end{center}
\end{titlepage}


\tableofcontents
\markboth{Contents}{}

\newcommand{\sectionbreak}{%
  \par%
  \begin{center}---\texttt{*}---\end{center}%
  \clearpage%
}%

\section{Introduction}

Reader with advanced tips and tricks for Stata.
I'll add some introductory text about the reader here.
Incomplete sections will have a very short description of what will be described in them.
Could include some information about the thesis project, AEclipse, and maybe myself.
I assume some knowledge of and/or experience with both Stata and programming in general.
Comments and feedback are always welcome.

\section{Information Management}

Section for some ``basic'' commands that are either a prerequisite for programming or make it much easier.

\subsection{help}

help command

\subsection{scalars and matrices}

storing information in scalars and matrices

\subsection{display}

display command

\subsection{macros}

globals and locals

\subsection{loops}

different types of loops

\section{Custom Commands}

Section on how to write a custom command.

\subsection{program}

defining a program

\subsection{arguments}

adding user input to a program

\subsection{syntax}

adding standard Stata syntax to a program

\subsection{temporary variables}

using temporary variables

\subsection{output}

defining program output

\section{General Tips}

Things I found helpful in using Stata.
%\printbibliography

\end{document}
