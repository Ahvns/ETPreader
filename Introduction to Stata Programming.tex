% document class settings
\documentclass[12pt, a4paper]{article}

% encoding
\usepackage[utf8]{inputenc}
\usepackage[T1]{fontenc}

% opmaak
\usepackage{microtype}
\usepackage{XCharter}
\usepackage[T1]{eulervm}
\usepackage[margin=2.5cm]{geometry}

\usepackage{titlesec}

% regelafstand
\usepackage{setspace}
\onehalfspacing%

% header & footer
\usepackage{fancyhdr}
\pagestyle{fancy}
\fancyhead[L]{\small\itshape~Introduction to Stata Programming}%
\fancyhead[R]{\small\itshape\leftmark~}%
\fancyfoot[C]{\thepage}%
\setlength{\headheight}{14pt}%
\renewcommand{\sectionmark}[1]{%
  \markboth{#1}{}}%

% bibliography
\usepackage[style=apa,natbib]{biblatex}
\addbibresource{references.bib}


% hyperlinks
\usepackage{hyperref}
\hypersetup{
  colorlinks=true,
  allcolors=black,
  linkcolor=blue,
  urlcolor=blue,
  citecolor=blue,
  linktocpage=true,
  hypertexnames=false
}
% easy references to tables/figures etc, not sure yet if necessary
\usepackage[capitalise,noabbrev]{cleveref}

% less "hanging" text at bottom or top of pages
\widowpenalty=10000
\clubpenalty=10000


\begin{document}

\begin{titlepage}\thispagestyle{empty}
    \begin{center}
        \vspace*{0.5cm}
        \LARGE
        Introduction to Stata Programming\\
        \Large
        Erasmus Thesis Project\\
        \vspace{1cm}
        \large
        Armin Hoendervangers\\
        AEclipse ETP Committee\\

        \vspace{2cm}
        Current version: \today\footnote{This is a work in progress, so the document will have incomplete sections and may contain mistakes.}

        \href{https://github.com/Ahvns/ETPreader/raw/main/Introduction%20to%20Stata%20Programming.pdf}{Click here for latest version}

        \vfill
    \end{center}
\end{titlepage}


\tableofcontents
\markboth{Contents}{}

\newcommand{\sectionbreak}{%
  \par%
  \begin{center}---\texttt{*}---\end{center}%
  \clearpage%
}%

\section{Introduction}

In this reader you'll find my best attempt at showing how to start programming in Stata.
While writing my bachelor thesis,
I took it upon myself to figure out how to write my own commands,
hoping it would allow me to save some time coding.
Unfortunately, it took much more time than I could have expected in advance,
but it has been worth the effort;
I'm now able to code much more efficiently in Stata and seem to have earned a reputation as a ``programming guy'' in my master's specialisation.

Last year I took part in all of AEclipse's Erasmus Thesis Project (\textsc{etp}) events,
which have been a huge help in writing what I believe to be a high quality thesis --
I even ended up winning the competition.
Joining this year's organising committee is my attempt at ``giving back'',
as it were,
and so is this reader.
As learning how to program in Stata has been a bit of a rabbit hole,
I hope to make this process somewhat easier and more streamlined for those of you that are also interested in this.
For those that are not \emph{that} interested in programming,
I hope you'll still find some useful tips and tricks scattered throughout the reader.

Throughout the reader,
I assume some knowledge of and/or experience with both Stata and programming in general.
As I've been working on this reader next to my own studies,
I haven't spend as much time on it as I would have liked in advance.
There could very well be some mistakes or incomplete bits in the reader due to this;
if you come across any, feel free to let me know and I'll do my best to fix it \textsc{asap}!
In general, any questions, comments,
or feedback for this reader is more than welcome,
and can be sent to \href{mailto:thesisproject@aeclipse.nl}{thesisproject@aeclipse.nl}.
If you're looking for more information on \textsc{etp}, look no further than AEclipse's \href{https://www.aeclipse.nl/etp}{website}.
Finally, I hope the reader is useful for you and I wish you the best of luck in writing your thesis!

\paragraph{Example code}
This reader contains several pieces of example code.
All code has been written so that it can be copy-pasted to a do-file and run using Stata as is,
unless specified otherwise.
All example code is also available as separate do-files on the reader's GitHub page \href{https://github.com/Ahvns/ETPreader/tree/main/Example%20do-files}{here}.
Note that some of the code in the reader contains automatically generated linebreaks,
which might be included if you copy and paste the code into a do-file yourself.
If any of the code doesn't work,
please check if any command is broken up into multiple lines where it should not be and fix this.
If the code still doesn't work, I've likely made a mistake --
please let me know if this happens!

\paragraph{Acknowledgements}
First of all,
I'd like to thank \dots for his enthusiasm for and support of my idea of creating this reader. % ask if he's okay with being named
His feedback has been very helpful,
and it's been great organising \textsc{etp} together.

% thanks to those giving workshop -- ask if they are okay with being named here


\section{Information Management}


Information is incredibly important in programming,
whether it is what a command does, how its used,
or what is contained in variables.
I take information as a starting point,
and this section is all about obtaining and storing various forms of information.


\subsection{The \texttt{help} command}

Perhaps the most important command in Stata,
\texttt{help} allows quick access to information on \emph{any} command Stata has.
The help-files Stata provides are incredibly detailed,
including information on how to use the command (its \emph{syntax}),
what the command does, its output, examples,
and sometimes even the theory behind it.
As useful as it is, the help-files might seem daunting at first.
Understanding their structure is key to (quickly) obtain information without falling into despair.
I'll highlight what I believe to be the most important parts of the help-files through an example.

If I type \mintinline{stata}{help summarize},
Stata opens the window in \cref{fig:hlpsum}.
In help-files, the typography on its own already gives us a lot of information.

\textbf{Bold} words indicate commands or options; if we want to use these,
we type them exactly as they are written down.
In our case,
\textbf{\texttt{summarize}} is written in bold under the syntax heading.
It is, as we know, indeed a command.

\textit{Italicised} text indicates something that should be substituted.
Here, \textit{\texttt{varlist}} tells us that we should write down a list of variable names here -- should we want to use this option.

Optional arguments and functions are indicated by being [in brackets].
This means that anything that is written within brackets in the syntax is something that does not have to be specified for a command to work.

\underline{Underlined} text indicates the \emph{minimum} abbreviation of a command or option.
In the case of \mintinline{stata}{summarize},
I could simply write \mintinline{stata}{su}.
Additional letters are also allowed and how to use this is mostly personal preference.
Personally, I always write \mintinline{stata}{sum} as its clearer to me what that means than \mintinline{stata}{su} would,
but it still saves me the time and space from writing the command out in its entirety.
When abbreviating commands,
make sure you are familiar enough with them to remember what an abbreviation means if you open your do-files one week later.
Having to look it up every time you see an abbreviation can be quite a pain.

Finally, any text in \textcolor{blue}{blue} is a hyperlink,
generally leading to more information on whatever is written down.\footnote{~Note that the exact colour depends on Stata's colour scheme, but the default and dark schemes do use blue.}

\begin{figure}[tbp]\centering
  \caption{Help-file for \texttt{summarize}}\label{fig:hlpsum}
  \vspace{1ex}
  \includegraphics[width=0.85\textwidth]{helpsummarize}
\end{figure}

\subsection{Scalars and matrices}

Rather than storing information in variables,
Stata offers us a couple of different ways of storing information independently of a datset.
Scalars and matrices are perhaps the most basic of these options.
Other than the names suggest, scalars are not limited to containing only numbers:
we are free to store other types of information such as strings in them as well.
Matrices can only contain numerical values, on the other hand.
Another difference between the two is the amount of information stored:
a scalar contains a single piece of information,
whereas a matrix can contain multiple pieces of information.

Scalars can be created using the \mintinline{stata}{scalar define} command,
although the \mintinline{stata}{define} can be left out.
For example,
if I want to create a scalar containing the number 7,
I could type:
\begin{minted}{stata}
  scalar define number = 7
\end{minted}

After running this line of code -- either through a do-file or the Stata console -- Stata has now created a scalar with the name ``number'' that contains the numerical value 7.
Naturally, we cannot always remember every piece of information we have stored.
If we want to know what scalars currently exist in Stata's memory, we use another command:
\begin{minted}{stata}
  scalar define number = 7
  scalar define second = "two"
  scalar list
\end{minted}
After running the second command, Stata returns us a list of all scalars with both their name and value:
\small\begin{verbatim}
  . scalar list
      second = two
      number =          7
\end{verbatim}\normalsize
Note that we can also type \mintinline{stata}{dir} instead of \mintinline{stata}{list} and obtain the same result.

Compared to scalars, matrices are both more versatile and more complicated to work with.
As they store multiple pieces of information, every piece of information also needs a position.
Creating a matrix is slightly different compared to creating a scalar:
\begin{minted}{stata}
  matrix input numbers = ( 7 , 2 \ 1 , 2)
\end{minted}
This creates a two by two matrix (i.e.\ two rows and two columns),
containing the values 7 and 2 in the first row and the values 1 and 2 in the second row.
In this command, commas seperate row values while backslashes start a new row.
Again, \mintinline{stata}{input} can be omitted when creating a matrix.
There is also a \mintinline{stata}{matrix define} command,
but this is used when we do computations with already existing matrices.
\mintinline{stata}{input} is used when inputting matrices by hand.

To see existing matrices, we use two commands:
\begin{minted}{stata}
  matrix input numbers = ( 7 , 2 \ 1 , 2)
  matrix dir
  matrix list numbers
\end{minted}
The first of these shows us a list of all matrices in Stata's memory and their size,
while the second command shows us the values of the matrix called ``numbers'':
\small\begin{verbatim}
  .   matrix dir
        numbers[2,2]

  .   matrix list numbers

  numbers[2,2]
      c1  c2
  r1   7   2
  r2   1   2
\end{verbatim}\normalsize

Finally,
we can remove scalars and matrices from Stata's memory using their respective command followed by \mintinline{stata}{drop}:
\begin{minted}{stata}
scalar drop _all
matrix drop numbers
\end{minted}
We can remove a specific scalar or matrix by specifying its name,
or we can remove all existing scalars or matrices by typing \mintinline{stata}{_all} instead of a name.
Scalars and matrices are also removed from memory when you close Stata itself or when you issue the \mintinline{stata}{clear all} command.

The \mintinline{stata}{help} file for both commands provide a lot more information on both scalars and matrices, especially so for the latter.

\subsection{macros}

globals and locals

\subsection{display}

display command


\section{Custom Commands}


Section on how to write a custom command.

\subsection{program}

defining a program

\subsection{arguments}

adding user input to a program

\subsection{syntax}

adding standard Stata syntax to a program

\subsection{temporary variables}

using temporary variables

\subsection{output}

defining program output


\section{General Tips}

Things I found helpful in using Stata.
%\printbibliography

\end{document}
