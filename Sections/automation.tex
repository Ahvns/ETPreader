
In this section we'll go over several commands that can be very useful for automating certain bits of code.

\subsection{Grouped Command Execution}
One of the easiest ways we can repeat a certain command for different groups of observations is with the \st{bysort} prefix.
This prefix lets us run the command we use it with for every group defined by a variable separately.
In my experience it's mostly useful for generating variables in programming,
but it can also be used as a quick and dirty way to compare variables across groups.
We can use the prefix like so: \st{bysort varlist: command},
where \st{varlist} is a list of the variables -- or a single variable --identifying the different groups,
and \st{command} is the command we would like to run.
Note that \st{bysort varlist:} is equivalent to using \st{by varlist, sort:}.
The \st{by} prefix does not work without sorting the data, so it is generally easier to just use \st{bysort}.
\cref{lst:sort} provides an example.

\begin{listing}[htp]
\caption{bysort.do}\label{lst:sort}
\inputst{bysort.do}
\end{listing}

\subsection{Conditionals}
Conditionals, or if-statements, are where the real fun stuff begins.
To put it simply,
they allow us to differentiate our code based on anything we can turn into an expression that evaluates to true or false.
Stata recognises two types of if-statements: one as a command suffix,
and one for programming.

\subsubsection{Command suffix if}
The command suffix if statement is the simpler of the two.
By adding an if-statement to the end of a command (but before the options!) we tell Stata to only use the specified subset of our data.
Suppose we have some individual-level data with a dummy variable \st{female} indicating an individual's gender, i.e. it is 1 for females and 0 for males.
We can then tell a command, such as \st{summarize},
to only use the observations for female individuals by typing \st{sum varlist if female == 1}.
As we're using a dummy variable, we could even shorten this to \st{sum varlist if female}!
This is because our expression, \st{female == 1},
evaluates either to true or false for each observation.
A variable containing either true or false is also known as a \emph{boolean} and is generally stored as either a value of 1 (true) or 0 (false).
Dummy variables are structured in the same way:
they have a value of 1 if something is true,
such as an individual's gender being female,
or 0 if false.
We can thus exploit this to make much more compact if-statements.

We can also use non-binary variables for our if-statements,
which allows us to use the full breadth of Stata's relational operators.
Furthermore, we can combine several if-statements using logic operators,
such as and, not, and or.
For the specific characters of each operator, see \st{help operator}.
The key takeway here is that any expression following \st{if} will evaluate to either true (1) or false (0).
If the statement evaluates to true for an observation it is used,
and if it evaluates to false it is left out.

On this note,
a quick tip for generating dummy variables.
Often people (including past me) do this in a roundabout way:
\begin{minted}{stata}
generate dummy = 1 if expression
replace dummy = 0 if dummy == . // missing
\end{minted}
As any if-statement already evaluates to either 1 or 0,
it is much simpler (and cleaner) to write:
\begin{minted}{stata}
generate dummy if expression
\end{minted}
If you work with a lot of dummy variables, this will save a lot of time!

\subsubsection{Programming if}
Stata's programming if-statements have a multitude of uses.
They allow us to execute bits of code only if a specified expression is true.
For a single command,
the syntax is as follows:
\begin{minted}{stata}
if expression command
\end{minted}
where the expression works in the same way as in the suffix if-statement.
In programming if-statements,
you probably won't be using variable names,
but rather locals or scalars that you defined previously.
Of course,
you are free to do as you like:
locals and scalars can also be used in suffix if-statements,
and variables can be used in programming if-statements,
although the latter will generally include an observation number to identify a specific value.\footnote{I.e.\ by writing \st{varname[number]} instead of simply \st{varname}}

We can also include multiple commands in a single if-statement.
The syntax for this is slightly different:
\begin{minted}{stata}
if expression {
    command1
    command2
    ...
}
\end{minted}
Should the expression evaluate to true,
Stata will execute all commands enclosed by the curly brackets.
Note that no command may follow the opening curly bracket (comments are fine) in the same line,
and the closing curly bracket must have a line just for itself.
While not necessary,
I recommend indenting the commands inside a code block like this to keep your code organised.

After a programming if-statement,
we can follow up with \st{else}.
The code following an else-statement executes when the if-statement is evaluated to be false.
An else-statement is written in the same way as an if-statement, except no expression is given.
Using both looks like this:
\begin{minted}{stata}
if expression {
    commands // these are executed if true
}
else {
    othercommands // these are executed if false
}
\end{minted}

\subsection{loops}

different types of loops
