
In this section we'll go over several commands that can be very useful for automating certain bits of code.

\subsection{Grouped command execution}
One of the easiest ways we can repeat a certain command for different groups of observations is with the \st{bysort} prefix.
This prefix lets us run the command we use it with for every group defined by a variable separately.
In my experience it's mostly useful for generating variables in programming,
but it can also be used as a quick and dirty way to compare variables across groups.
We can use the prefix like so: \st{bysort varlist: command},
where \st{varlist} is a list of the variables -- or a single variable --identifying the different groups,
and \st{command} is the command we would like to run.
Note that \st{bysort varlist:} is equivalent to using \st{by varlist, sort:}.
The \st{by} prefix does not work without sorting the data, so it is generally easier to just use \st{bysort}.
\cref{lst:sort} provides an example.

\begin{listing}[htp]
\caption{bysort.do}\label{lst:sort}
\inputst{bysort.do}
\end{listing}

\subsection{Conditionals}


\subsection{loops}

different types of loops
