In this reader you'll find my best attempt at showing how to start programming in Stata.
While writing my bachelor thesis,
I took it upon myself to figure out how to write my own commands,
hoping it would allow me to save some time coding.
Unfortunately, it took much more time than I could have expected in advance,
but it has been worth the effort;
I'm now able to code much more efficiently in Stata and seem to have earned a reputation as a ``programming guy'' in my master's specialisation.

Last year I took part in all of AEclipse's Erasmus Thesis Project (\textsc{etp}) events,
which have been a huge help in writing what I believe to be a high quality thesis --
I even ended up winning the competition.
Joining this year's organising committee is my attempt at ``giving back'',
as it were,
and so is this reader.
As learning how to program in Stata has been a bit of a rabbit hole,
I hope to make this process somewhat easier and more streamlined for those of you that are also interested in this.
For those that are not \emph{that} interested in programming,
I hope you'll still find some useful tips and tricks scattered throughout the reader.

Throughout the reader,
I assume some knowledge of and/or experience with both Stata and programming in general.
As I've been working on this reader next to my own studies,
I haven't spend as much time on it as I would have liked in advance.
There could very well be some mistakes or incomplete bits in the reader due to this;
if you come across any, feel free to let me know and I'll do my best to fix it \textsc{asap}!
In general, any questions, comments,
or feedback for this reader is more than welcome,
and can be sent to \href{mailto:thesisproject@aeclipse.nl}{thesisproject@aeclipse.nl}.
If you're looking for more information on \textsc{etp}, look no further than AEclipse's \href{https://www.aeclipse.nl/etp}{website}.
Finally, I hope the reader is useful for you and I wish you the best of luck in writing your thesis!

\paragraph{Example code}
This reader contains several pieces of example code.
All code has been written so that it can be copy-pasted to a do-file and run using Stata as is,
unless specified otherwise.
All example code is also available as separate do-files on the reader's GitHub page \href{https://github.com/Ahvns/ETPreader/tree/main/Example%20do-files}{here}.
Note that some of the code in the reader contains automatically generated linebreaks,
which might be included if you copy and paste the code into a do-file yourself.
If any of the code doesn't work,
please check if any command is broken up into multiple lines where it should not be and fix this.
If the code still doesn't work, I've likely made a mistake --
please let me know if this happens!

\paragraph{Acknowledgements}
First of all,
I'd like to thank \dots for his enthusiasm for and support of my idea of creating this reader. % ask if he's okay with being named
His feedback has been very helpful,
and it's been great organising \textsc{etp} together.

% thanks to those giving workshop -- ask if they are okay with being named here
